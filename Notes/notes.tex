\documentclass[12pt, titlepage]{article}
\usepackage{amssymb, amsmath, hyperref,setspace,physics}
\usepackage{color,parskip,siunitx}
\usepackage{epsfig, graphicx, todonotes}
\usepackage{verbatim}

\begin{document}
I need to decide what to focus on in this work. There is a lot of information out there about the laser, different regimes. Do I fully derive the cold-fluid plasma equations
\section{Cold Fluid Equations}

In treating a plasma mathematically as a fluid, we must consider that the fluid particles
will also have to obey the Maxwell equations. This means that the well known Navier-Stokes
equation has to be modified to include additional terms. In a simple model, where the electron
motion is confined to motion in the $x$ direction, and $T=0$ and $B=0$, the full equations
needed to describe the cold plasma is given by:
\begin{subequations}
    \begin{align}
        \div{\vb{E}} & = 4 \pi e \qty(n_i-n_e) \\
    m n_e \left( \pdv{\vb{v}_e}{t} \right) &= -e n_e \vb{E} \\
        \pdv{n_e}{t} + \div{ \qty(n_e \vb{v}_e)} & = 0\\
    \end{align}
\end{subequations}

With an small incident electric field, we can linearize the equations about
equationbrium:
\begin{align*}
    &\vb{E} \rightarrow \vb{E_0} + \vb{E_1}\\
    &n \rightarrow 0 + \delta n \\
    &\vb{v} \rightarrow \vb{v_0} + \vb{v_1}
\end{align*}

Only keeping terms to first order, and assuming that the perturbations vary sinusoidally,
this gives us the disperion of Langumir waves in a cold plasma. 

\begin{equation}
\label{eq:dispersion}
\omega = \qty(\frac{ne^2}{m_e \epsilon})^{1/2} = \omega_p,
\end{equation}
with $e$ as the electron's charge, $m_e$ the electron's mass, $n$ is the number
density of electrons in the medium, and $\epsilon$ is the permativity of the plasma.

\section{Laser-Driven Plasma Waves for Particle Acceleration and X-Ray Production}
The nonlinear case is where $a_0 = e\vb{A}(m_e c^2)^{-1} > 1$. Transfering to
a frame moving at the same rate as the phase veloctiy of the packet,  $t\rightarrow \tau$, $z-v_g t \rightarrow \xi $, and assuming that the pulse 
doesn't evolve, we can get Poisson's equation:
\begin{equation}
    \pdv[2]{\phi}{\tau} = k_p^2 \qty( \frac{n_e}{n_{e0}} - 1)
\end{equation}
        \todo{still need to be able to derive this \ldots}
The relativisitic factor then becomes:
\begin{equation}
    \gamma = \frac{\gamma_\bot^2}{1-\beta_z^2} 
\end{equation}
I think I may just show a figure for the non-linear case. The equation is hard 
to look at and understand. Cite heavily.


\section{Laboratory Visualization of Laser-Driven Plasma Accelerators in the BUbble Regime}
\section{Injection in Plasma-Based Electron Accelerators}
\section{Experimental Programs}
\section{Texas}
Should include figure from the Texas people. 

Using chirped-pulse-amplification (CPA), extremely high peak power can be achieved by modern ultra-fast laser. CPA works by first creating an ultrashort
pulse using a mode-locked laser (such as Ti:Sapphire). The pulse energy is fairly small to begin with\footnote{Typically around \SI{e-9}{\joule}, with duration $\tau =$ \SIrange{e-12}{e-14}{\second}, for peak powers of \SIrange{e3}{e5}{\watt} }. To avoid damage to the amplification stage, the pulse is given a `chirp' in the frequency domain: the simplest example of this is to send a pulse down an optical fibre, where the dispersion will naturally broaden the pulse
in time. Once the pulse has been stretched out, the peak power will decrease, and the pulse can be safely sent through a laser-amplification stage, typically with a gain between 2-100 per pass. Once the pulse has been safely amplified, it can be re-compressed, using a technology such as parallel grating compressor.
The whole process can increase the peak power from \SIrange{e3}{e14}{\watt}, as
in the Texas PetaWatt Laser at UT-Austin.

With laser peak power this high, it can easily ionize a solid target-- creating
the plasma that it will be propogating through. In the Texas setup, the laser pulse is incident on a gas-cell, where it will create the co-propogating density wave by `blowing out' the electrons from the region of the laser pulse.

This will set-up the longitudinal electric field that acts to accelerate the electrons. The electrons will be accelerated for the duration of the glass cell, and then be expelled into free space. 

There is an array of sensors to measure the incoming electrons. First, they pass
through a magnetic field, which bends them with a radius of curvature $ r = mv/(eB)$ \todo{make this relativisitic}. The slower moving electrons will be bent more, hitting an imagining plate. The faster moving electrons will continue on, largely undeflected. They are incident on two imagining plates: the first is a high-sensitivity plate that picks up medium energy electrons ($E >\SI{0.5}{\giga\electronvolt}$) that have been energery dispersed (not well collumate), the
second imagining plate is the one that picks up the well-collumated electron beam at $E \approx \SI{2}{\giga\electronvolt}$. Additionally, tungsten posts are placed at various places downsteam of the exit aperature. These posts will
cast shadows on the imaged plates, and simple geometry can be used to reveal
the trajectory of the electron beam.

\section{SLAC}

\section{Europe}

\section{Future Work}

The initial results in 2004 by [][][] where very promising, and progress is being made at an accelerated rate. However, it is still difficult to understand the minutia of the dynamics of LPA. This is due to the highly non-linear nature of the process. To make further progress, advances need to be made in three catagories: first, finer control over the laser parameters needs to be established, the pulse duration, pulse shape, pulse energy; second, better imaging and
diagnostic technologies need to be developed so that the pulse can be imaged in situ; third, more work needs to be done on the theoretical side so that experimentalists have an idea of what knobs to twist.
\end{document}
