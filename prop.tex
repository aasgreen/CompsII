\documentclass[12pt]{article}
\usepackage{amssymb, amsmath, hyperref,setspace}
\usepackage[margin=1in]{geometry}
\usepackage{epsfig, graphicx}
\usepackage{verbatim}

\begin{document}
\bibliographystyle{plain}

\centerline{ \huge Laser-Wake Fields}
\bigskip

The most common form of electron accelator can only sustain acceleration gradients of a few $\frac{\textrm{MeV}}{m}$\cite{faure2004laser}, to get the energies needed (GeV's) for most useful applications this requires that they are of considerable size and expense. 
    
    Based on work done by \textit{Tajima and Dawson}\cite{PhysRevLett.43.267}, a new form of accelerator is being developed, based on the high electromagnetic field gradients that can be sustained in plasmas. Because these gradients are orders of magnitude higher than conventional accelerators, the size and therefore cost can be reduced manifold. This reduction would lead to true tabletop sources of high energy electrons-- something useful from a applications as
    diverse as medical technology to high-energy physics\cite{malka2008principles}.


A surfer ``hanging ten'' on a wave is an iconic image, one which, on the surface might not have much to do with relativistic electron acceleration-- yet serves as a very useful analogy for the physics occuring in these plasma accelerators.

A high-intensity laser pulse is shot at a plasma. The electromagnetic poneramotive force tears the electrons away, until a static electric field produced by an density imbalance between the heavier positive ions and the more mobile, lighter electrons will pull the electrons back towards the ions. This sets up a plasma oscillation, that travels after the laser pulse, much like the wake of a boat.

Electrons with the proper momentum can catch this `wake', riding it much like the surfer would an ocean wave. Except, in this case, the electron can be accelerated to near light speed.\cite{LPOR:LPOR200810062}.


The proposed review paper will review the state-of-the-art in laser-wake field electron acceleration.
It will begin with a theoretical review of the physical phenomenon of laser-wake fields, studying the interesting dynamics of the emergent light-matter interaction. Then, with a more narrow focus on laser-wake fields as an electron accelerating technology and x-ray source, this paper will discuss the current technical challenges facing the field, as well as highlighting the current research
avenues being followed to overcome those challenges.
\bibliography{../pap/rep.bib}
\end{document}
