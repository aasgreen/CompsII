\documentclass[12pt, titlepage]{article}
\usepackage{amssymb, amsmath, hyperref,setspace}
\usepackage[margin=1in]{geometry}
\usepackage{color,parskip,siunitx,physics}
\DeclareMathAlphabet{\mathscr}{OT1}{pzc}{m}{it}
\usepackage{epsfig, graphicx}
\usepackage{verbatim}

% for double partial derivatives
\title{Laser-Wakefield Electron Accelerators}
\author{Adam A. S. Green}
\begin{document}
\bibliographystyle{plain}
\maketitle
\tableofcontents
\begin{abstract}
In the late 1970's, Tajima and Dawson proposed a method of acceleration electron
using the large electric field gradients that plasmas are capable of sustaining. They showed that when a large amplitude laser pulse is sent through an plasma, it is capable of creating a co-propogating longitudinal plasma wave that is capable of having an electric field gradient of multiple Gev/cm. 

In this work, we review the state-of-the-art experimental progress being made in creating a fully functioning electron accelerator based on laser plasma acceleration (LPA) technology. We discuss how the energy in a  transverse, oscillitory electric field of a high intensity pulsed laser can be converted into an electric field gradient that can be used to accelerate electrons, and we will briefly review the physics of its propogation. 
\end{abstract}
\section{Introduction}
\label{sec:intro}
The ability to accelerate electrons to relativistic energies ($ E > m_e c$) has had of the the largest impacts in modern science: leading to advances in particle physics, radiology, and condensed matter.
It remains one of the most systematic ways to produce collumated x-ray's.
Unfortunately, the maximum accelerating gradient able to be produced by current tecnology is on the scale of $\sim \SI{100}{\electronvolt.cm^{-1}}$, which, in order to produce relativistic
energies ($\sim \SI{.511}{\mega\electronvolt}$), requires a lot
of centimeters: CERN has a circumfrence of \SI{27}{km}. The large size needed has precluded the use widespread use of relativistic electrons.

In contrast, the accelerating gradient in typical LPA technology is $\sim \SI{1}{\giga\electronvolt.cm^{-1}}$. This then is the promise of LPA technology: table-top sized experiments that will allow the widespread disemmenation of relativistic electrons. 
\section{The Physics of Laser-Plasma-Acceleration}

\subsection{The Need for Plasmas}
The reason that LPA has had such a resurgance is that it was only recently that laser technology was able to produce the electric field gradients neccesary. For instance, the laser used at UT-Austin in their LPA technology uses required a peak intensity of \SI{e18}{W.cm^{-2}} {\bf (not quite right, this is another group).}

It is tempting to ask why the plasma is needed at all when you have a laser capable of generating such high power-- why not just use the laser? In {\bf date} Woodward and Lauren proved that laser's cannot actually be used to accelerate particles, for much the same reason that boats only move up and down in linear ocean waves. {\bf maybe not the full story-- the ponderamotive force can't?}

The plasma acts as an intermediary, allowing the large electronic gradients present in a propogating high-intensity laser pulse to be transfered into a co-propogating longitudal wave. It is this longitudal wave that allows the electrons to be accelerated. In the section that follows, we will dicuss the physics behind this energy transfer in more depth.
\section{The Creation and Propogation of Plasmons and Bubbles}
The full theory required to describe the current state-of-the-art LPA technology is
too difficult to treat analytically, so the majority of theoretical research currently
being done in the field uses numerics. However, the basic physics involved with LPA technology
can be illustrated quite simply, which can be useful to build an intuition about the
processes that occur. In the following sections, we will review a basic physical model
of the laser plasma interaction.
\subsection{Linear Regime: Cold Fluid Equations}
To first order the plasma can be treated as a set of uncoupled harmonic oscillators. When the high-intensity, short-pulsed laser is incident, it will have two main actions: the first is to accelerate\footnote{The term used to describe this in plasma literature is `quiver', as the time-average motion amounts to zero because of the oscillitory nature of the radial electric field.l}
electrons along the radial electric field; the second is to accelerate electrons along the propogation direction due to the ponderomotive force. 

The ponderomotive force is due to the gradient in electromagnetic energy along a pulsed laser. {\bf show gaussian picture}. It will tend to create a `bubble', as it expels the lighter electrons from the laser pulse, leaving the heavier ions.

In our model, the ponderomotive force will act as a driving force on the plasma,
which has a natural oscillation rate $\omega_p$. 
\begin{equation}
    \label{eq:linear density oscillation}
    \qty(\pdv[2]{}{t} +\omega_p^2) \frac{\delta n_e}{n_{e0}} = c^2 \laplacian{ \frac{a^2}{2}}
\end{equation}

It is this density oscillation which creates an electric field through poisson's equation: $\vb{k}\dotproduct\vb{E} =\delta n_e/\epsilon$. It is this
longitudinally propogating electric field that will accelerate the electrons in the plasma.
 
{\it still need to mention resonance}

The propogating laser packet expels electrons from its local region, which acts
as a driving force to create a co-propogating density oscillation. This creates in effect a cavity of ions at moves at the phase velocity of the laser packet. 
This gives rise to an electric field gradient that will be able to accelerate electrons.

\subsection{Nonlinear-1D}

\subsection{Injecting Electrons into Bubbles}

\section{Experimental Set-Up and State-of-the-Art}

\section{Future Work and Outlook for the Field}

\section{Conclusion}
\end{document}

